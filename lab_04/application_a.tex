\chapter*{Приложение А}
\addcontentsline{toc}{chapter}{Приложение А}

\begin{lstlisting}[label=lst:code_1,caption=Содержимое файла host\_example.cpp]
// This is a generated file. Use and modify at your own risk.
////////////////////////////////////////////////

/***********************************************
Vendor: Xilinx
Associated Filename: main.c
#Purpose: This example shows a basic vector add +1 (constant) by manipulating
#         memory inplace.
***********************************************/

#include <fcntl.h>
#include <stdio.h>
#include <iostream>
#include <stdlib.h>
#include <string.h>
#include <math.h>
#ifdef _WINDOWS
#include <io.h>
#else
#include <unistd.h>
#include <sys/time.h>
#endif
#include <assert.h>
#include <stdbool.h>
#include <sys/types.h>
#include <sys/stat.h>
#include <CL/opencl.h>
#include <CL/cl_ext.h>
#include "xclhal2.h"

////////////////////////////////////////

#define NUM_WORKGROUPS (1)
#define WORKGROUP_SIZE (256)
#define MAX_LENGTH 8192
#define MEM_ALIGNMENT 4096
#if defined(VITIS_PLATFORM) && !defined(TARGET_DEVICE)
#define STR_VALUE(arg)      #arg
#define GET_STRING(name) STR_VALUE(name)
#define TARGET_DEVICE GET_STRING(VITIS_PLATFORM)
#endif

////////////////////////////////////////

cl_uint load_file_to_memory(const char *filename, char **result)
{
	cl_uint size = 0;
	FILE *f = fopen(filename, "rb");
	if (f == NULL) {
		*result = NULL;
		return -1; // -1 means file opening fail
	}
	fseek(f, 0, SEEK_END);
	size = ftell(f);
	fseek(f, 0, SEEK_SET);
	*result = (char *)malloc(size+1);
	if (size != fread(*result, sizeof(char), size, f)) {
		free(*result);
		return -2; // -2 means file reading fail
	}
	fclose(f);
	(*result)[size] = 0;
	return size;
}

int main(int argc, char** argv)
{
	
	cl_int err;                            // error code returned from api calls
	cl_uint check_status = 0;
	const cl_uint number_of_words = 4096; // 16KB of data
	
	
	cl_platform_id platform_id;         // platform id
	cl_device_id device_id;             // compute device id
	cl_context context;                 // compute context
	cl_command_queue commands;          // compute command queue
	cl_program program;                 // compute programs
	cl_kernel kernel;                   // compute kernel
	
	cl_uint* h_data;                                // host memory for input vector
	char cl_platform_vendor[1001];
	char target_device_name[1001] = TARGET_DEVICE;
	
	cl_uint* h_axi00_ptr0_output = (cl_uint*)aligned_alloc(MEM_ALIGNMENT,MAX_LENGTH * sizeof(cl_uint*)); // host memory for output vector
	cl_mem d_axi00_ptr0;                         // device memory used for a vector
	
	if (argc != 2) {
		printf("Usage: %s xclbin\n", argv[0]);
		return EXIT_FAILURE;
	}
	
	// Fill our data sets with pattern
	h_data = (cl_uint*)aligned_alloc(MEM_ALIGNMENT,MAX_LENGTH * sizeof(cl_uint*));
	for(cl_uint i = 0; i < MAX_LENGTH; i++) {
		h_data[i]  = i;
		h_axi00_ptr0_output[i] = 0;
		
	}
	
	// Get all platforms and then select Xilinx platform
	cl_platform_id platforms[16];       // platform id
	cl_uint platform_count;
	cl_uint platform_found = 0;
	err = clGetPlatformIDs(16, platforms, &platform_count);
	if (err != CL_SUCCESS) {
		printf("ERROR: Failed to find an OpenCL platform!\n");
		printf("ERROR: Test failed\n");
		return EXIT_FAILURE;
	}
	printf("INFO: Found %d platforms\n", platform_count);
	
	// Find Xilinx Plaftorm
	for (cl_uint iplat=0; iplat<platform_count; iplat++) {
		err = clGetPlatformInfo(platforms[iplat], CL_PLATFORM_VENDOR, 1000, (void *)cl_platform_vendor,NULL);
		if (err != CL_SUCCESS) {
			printf("ERROR: clGetPlatformInfo(CL_PLATFORM_VENDOR) failed!\n");
			printf("ERROR: Test failed\n");
			return EXIT_FAILURE;
		}
		if (strcmp(cl_platform_vendor, "Xilinx") == 0) {
			printf("INFO: Selected platform %d from %s\n", iplat, cl_platform_vendor);
			platform_id = platforms[iplat];
			platform_found = 1;
		}
	}
	if (!platform_found) {
		printf("ERROR: Platform Xilinx not found. Exit.\n");
		return EXIT_FAILURE;
	}
	
	// Get Accelerator compute device
	cl_uint num_devices;
	cl_uint device_found = 0;
	cl_device_id devices[16];  // compute device id
	char cl_device_name[1001];
	err = clGetDeviceIDs(platform_id, CL_DEVICE_TYPE_ACCELERATOR, 16, devices, &num_devices);
	printf("INFO: Found %d devices\n", num_devices);
	if (err != CL_SUCCESS) {
		printf("ERROR: Failed to create a device group!\n");
		printf("ERROR: Test failed\n");
		return -1;
	}
	
	//iterate all devices to select the target device.
	for (cl_uint i=0; i<num_devices; i++) {
		err = clGetDeviceInfo(devices[i], CL_DEVICE_NAME, 1024, cl_device_name, 0);
		if (err != CL_SUCCESS) {
			printf("ERROR: Failed to get device name for device %d!\n", i);
			printf("ERROR: Test failed\n");
			return EXIT_FAILURE;
		}
		printf("CL_DEVICE_NAME %s\n", cl_device_name);
		if(strcmp(cl_device_name, target_device_name) == 0) {
			device_id = devices[i];
			device_found = 1;
			printf("Selected %s as the target device\n", cl_device_name);
		}
	}
	
	if (!device_found) {
		printf("ERROR:Target device %s not found. Exit.\n", target_device_name);
		return EXIT_FAILURE;
	}
	
	// Create a compute context
	//
	context = clCreateContext(0, 1, &device_id, NULL, NULL, &err);
	if (!context) {
		printf("ERROR: Failed to create a compute context!\n");
		printf("ERROR: Test failed\n");
		return EXIT_FAILURE;
	}
	
	// Create a command commands
	commands = clCreateCommandQueue(context, device_id, CL_QUEUE_PROFILING_ENABLE | CL_QUEUE_OUT_OF_ORDER_EXEC_MODE_ENABLE, &err);
	if (!commands) {
		printf("ERROR: Failed to create a command commands!\n");
		printf("ERROR: code %i\n",err);
		printf("ERROR: Test failed\n");
		return EXIT_FAILURE;
	}
	
	cl_int status;
	
	// Create Program Objects
	// Load binary from disk
	unsigned char *kernelbinary;
	char *xclbin = argv[1];
	
	//---------------------------------------
	// xclbin
	//---------------------------------------
	printf("INFO: loading xclbin %s\n", xclbin);
	cl_uint n_i0 = load_file_to_memory(xclbin, (char **) &kernelbinary);
	if (n_i0 < 0) {
		printf("ERROR: failed to load kernel from xclbin: %s\n", xclbin);
		printf("ERROR: Test failed\n");
		return EXIT_FAILURE;
	}
	
	size_t n0 = n_i0;
	
	// Create the compute program from offline
	program = clCreateProgramWithBinary(context, 1, &device_id, &n0,
	(const unsigned char **) &kernelbinary, &status, &err);
	free(kernelbinary);
	
	if ((!program) || (err!=CL_SUCCESS)) {
		printf("ERROR: Failed to create compute program from binary %d!\n", err);
		printf("ERROR: Test failed\n");
		return EXIT_FAILURE;
	}
	
	
	// Build the program executable
	//
	err = clBuildProgram(program, 0, NULL, NULL, NULL, NULL);
	if (err != CL_SUCCESS) {
		size_t len;
		char buffer[2048];
		
		printf("ERROR: Failed to build program executable!\n");
		clGetProgramBuildInfo(program, device_id, CL_PROGRAM_BUILD_LOG, sizeof(buffer), buffer, &len);
		printf("%s\n", buffer);
		printf("ERROR: Test failed\n");
		return EXIT_FAILURE;
	}
	
	// Create the compute kernel in the program we wish to run
	//
	kernel = clCreateKernel(program, "rtl_kernel_wizard_0", &err);
	if (!kernel || err != CL_SUCCESS) {
		printf("ERROR: Failed to create compute kernel!\n");
		printf("ERROR: Test failed\n");
		return EXIT_FAILURE;
	}
	
	// Create structs to define memory bank mapping
	cl_mem_ext_ptr_t mem_ext;
	mem_ext.obj = NULL;
	mem_ext.param = kernel;
	
	
	mem_ext.flags = 1;
	d_axi00_ptr0 = clCreateBuffer(context,  CL_MEM_READ_WRITE | CL_MEM_EXT_PTR_XILINX,  sizeof(cl_uint) * number_of_words, &mem_ext, &err);
	if (err != CL_SUCCESS) {
		std::cout << "Return code for clCreateBuffer flags=" << mem_ext.flags << ": " << err << std::endl;
	}
	
	
	if (!(d_axi00_ptr0)) {
		printf("ERROR: Failed to allocate device memory!\n");
		printf("ERROR: Test failed\n");
		return EXIT_FAILURE;
	}
	
	
	err = clEnqueueWriteBuffer(commands, d_axi00_ptr0, CL_TRUE, 0, sizeof(cl_uint) * number_of_words, h_data, 0, NULL, NULL);
	if (err != CL_SUCCESS) {
		printf("ERROR: Failed to write to source array h_data: d_axi00_ptr0: %d!\n", err);
		printf("ERROR: Test failed\n");
		return EXIT_FAILURE;
	}
	
	
	// Set the arguments to our compute kernel
	// cl_uint vector_length = MAX_LENGTH;
	err = 0;
	cl_uint d_scalar00 = 0;
	err |= clSetKernelArg(kernel, 0, sizeof(cl_uint), &d_scalar00); // Not used in example RTL logic.
	err |= clSetKernelArg(kernel, 1, sizeof(cl_mem), &d_axi00_ptr0);
	
	if (err != CL_SUCCESS) {
		printf("ERROR: Failed to set kernel arguments! %d\n", err);
		printf("ERROR: Test failed\n");
		return EXIT_FAILURE;
	}
	
	size_t global[1];
	size_t local[1];
	// Execute the kernel over the entire range of our 1d input data set
	// using the maximum number of work group items for this device
	
	global[0] = 1;
	local[0] = 1;
	err = clEnqueueNDRangeKernel(commands, kernel, 1, NULL, (size_t*)&global, (size_t*)&local, 0, NULL, NULL);
	if (err) {
		printf("ERROR: Failed to execute kernel! %d\n", err);
		printf("ERROR: Test failed\n");
		return EXIT_FAILURE;
	}
	
	clFinish(commands);
	
	
	// Read back the results from the device to verify the output
	//
	cl_event readevent;
	
	err = 0;
	err |= clEnqueueReadBuffer( commands, d_axi00_ptr0, CL_TRUE, 0, sizeof(cl_uint) * number_of_words, h_axi00_ptr0_output, 0, NULL, &readevent );

	if (err != CL_SUCCESS) {
		printf("ERROR: Failed to read output array! %d\n", err);
		printf("ERROR: Test failed\n");
		return EXIT_FAILURE;
	}
	clWaitForEvents(1, &readevent);
	
	// Check Results
	
	for (cl_uint i = 0; i < number_of_words; i++) {
		if ((h_data[i] * 4 - 16) != h_axi00_ptr0_output[i]) {
			printf("ERROR in rtl_kernel_wizard_0::m00_axi - array index %d (host addr 0x%03x) - input=%d (0x%x), output=%d (0x%x)\n", i, i*4, h_data[i], h_data[i], h_axi00_ptr0_output[i], h_axi00_ptr0_output[i]);
			check_status = 1;
		}
		//printf("i=%d, input=%d, output=%d\n", i,  h_axi00_ptr0_input[i] h_axi00_ptr0_output[i]);
	}
	
	
	//----------------------------------------
	// Shutdown and cleanup
	//----------------------------------------
	clReleaseMemObject(d_axi00_ptr0);
	free(h_axi00_ptr0_output);
	
	free(h_data);
	clReleaseProgram(program);
	clReleaseKernel(kernel);
	clReleaseCommandQueue(commands);
	clReleaseContext(context);
	
	if (check_status) {
		printf("ERROR: Test failed\n");
		return EXIT_FAILURE;
	} else {
		printf("INFO: Test completed successfully.\n");
		return EXIT_SUCCESS;
	}
	
	
} // end of main
	
\end{lstlisting}